\section{Program Chair Message}

Welcome to the 61st Annual Meeting of the Association for Computational Linguistics! We are excited to welcome everyone in Toronto.  

Most of the work of a program chair is behind the scenes: herding reviewers and chairs, wrangling data from various sources, and answering lots and lots of email. This is a volunteer position, so the only reward we get for this is our chance to make the process of submitting and reviewing papers to our conference better. This letter will outline some of those experiments.  

First, we asked reviewers for two scores: soundness and excitement.  Our goal was that any sound paper would be accepted to some ACL affiliated venue, but that the ``main conference'' distinction (limited by space) would be focused on the most exciting papers.  Our hope was that soundness would be less noisy than a single ``overall recommendation'' score, which would help reduce the randomness of decisions. Judging by the exit surveys, this change was well received: over 80\% of the chairs, reviewers and authors either expressed support or did not object to this change.

Next, we developed a new process for matching papers to reviewers based on keywords for not only the subject matter of the paper, but also its type of contribution and target language(s). This allowed more fine-grained control over the paper-reviewer matches, and we were also able to provide the chairs with context for the paper-reviewer matches.  

To improve review quality, we also updated the reviewer guidelines, and developed a system for the authors to flag specific types of issues with reviews. Finally, we have also proposed a new initiative for recognizing outstanding reviewers and chairs (73 awards at ACL'23).

Finally, we have tried to give more options for presentations.  Findings papers now have an in-person presentation spotlight slot and virtual posters in addition to recording videos.  Virtual posters have portals to link in-person attendees to virtual posters.  We have also brought back Miniconf and RocketChat to allow for better virtual communication between papers (regardless of where the authors are).

This conference is a result of the joint efforts of over ten thousand people. We deeply thank them all, and apologize for the many nagging emails we had to send out. In particular:

\begin{itemize}

\item the general chair Yang Liu, who led the whole process;
\item the incredible team of 70 SACs, 438 ACs, and 4490 reviewers, who were able to handle our record number of submissions;
\item the 13,658 authors for their phenomenal scientific contributions, which we were honored to shepherd through the reviewing process;
\item the ACL Executive (esp. Iryna Gurevych, Tim Baldwin, David Yarowsky, Yusuke Miyao, Emily M. Bender) for their support of many of our crazy ideas;
\item 21 ethics committee reviewers, chaired by Dirk Hovy and Yonatan Bisk, for their hard work to uphold the ACL code of ethics;
\item Our Best Paper Award committee (Jonathan Berant, Jose Camacho-Collados, Danqi Chen, Benjamin Van Durme, David Jurgens, Desmond Elliott, Sasha Luccioni, Jonathan May, Tom McCoy, Yusuke Miyao, Ekaterina Shutova, Emma Strubell, 
Jun Suzuki, Xiaojun Wan, Luke Zettlemoyer), who reviewed a record number of nominated papers under tight schedule; 
\item Our assistant Youmi Ma, for reducing our email and Softconf workload significantly and suggesting ideas to make the job run smoothly;
\item Past \*ACL PCs, including Smaranda Muresan, Preslav Nakov and 
Aline Villavicencio (ACL 2022), Yoav Goldberg, Zornitsa Kozareva, Yue Zhang (EMNLP 2022), Anna Rumshisky, Luke Zettlemoyer, Dilek Hakkani-Tur (NAACL 2021), for their advice and suggestions;
\item Publication chairs Ryan Cotterell, Chenghua Lin, Jesse Thomason, Lei Shu, and Lifu Huang, who ensured the proper formatting of camera-ready papers;
\item Emma Strubell, Ian Magnusson, and Jesse Dodge for their help in preparing publishable versions of Responsible NLP checklist; 
\item ACL Anthology director Matt Post;
\item TACL editors-in-chief (Asli Celikyilmaz, Roi Reichart, Ani Nenkova) and CL Editor-in-Chief Hwee Tou Ng for coordinating TACL and CL presentations with us;
\item Workshop chairs Annie Louis, Eduardo Blanco, and Yang Feng, for helping us to connect the Findings papers to possible presentation slots at workshops;
\item Rich Gerber at Softconf, who answered countless emails and implemented several new features on our request;
\item Our virtual infrastructure chairs (Pedro Rodriguez, Jiacheng Xu, Martín Villalba) and Underline team (Damira Mrsic, Sol Rosenberg) for enabling a new kind of hybrid experience, combining miniconf and Underline;
\item the ACL event director Jennifer Rachford and our visa support team (Ayana Niwa, Qingwen Liu, Renxiang Zhang, Samridhi Choudhary, and Tao You), who did everything possible to facilitate the Canada visa situation for ACL attendees.
\end{itemize}

% adding our own thanks on top of Yang's

%\section*{[TODO] Thank Yous}
%EACL 2023 is the result of a collaborative effort and a supportive community, and we want to acknowledge the efforts of so many people with whom we worked directly and made significant efforts in putting together the programme for EACL 2023! 
%\begin{itemize}
%\item Our General Chair, Alessandro Moschitti, who led the whole organising team, and helped with many of the decision processes;
%\item Our 43 Senior Area Chairs, who were instrumental in every aspect of the review process, from recruiting Area Chairs, correcting reviewer assignments, to making paper acceptances;
%\item Our 118 Area Chairs, who had the role of interacting with the reviewers, leading paper review discussions, and writing meta-reviews;
%\item The 1634 reviewers, who provided valuable feedback to the authors;
%The emergency reviewers, who provided their support at the last minute to ensure a timely reviewing process;
%\item Our Best Paper Selection Committee, who selected the best papers and the outstanding papers: Jonathan Kummerfeld (chair), Joakim Nivre, Bonnie Webber, Thamar Solorio and Hanna Hajishirzi;
%\item Our Ethics Committee, chaired by Zeerak Talat, for their hard work to ensure that all the accepted papers addressed the ethical issues appropriately, under a very tight schedule;
%\item Our amazing Publication Chairs, Carolina Scarton and Ryan Cotterell for compiling the proceedings in good time for the conference;

%\item Damira Mrsic from Underline, for her support in developing the virtual conference platform;
%\item Rich from Softconf, for tirelessly implementing our endless requests;
%\item Jennifer Rachford, who has worked tirelessly online and on-site to ensure that EACL 2023 is a success.
%\end{itemize}

%, and we look forward to seeing you in Toronto.


%After the last edition in 2021 having been held fully online due to the COVID pandemic, EACL 2023 is being held in “hybrid” mode this year, serving both virtual and in-person participants in Dubrovnik, Croatia. While the original plan was to hold the conference in Kyiv (which was the plan originally for EACL 2021), the ongoing war made the organisation in Ukraine impossible. In order to ensure that the original aim of strengthening the connections with the Ukrainian community is still served, our program features a dedicated session and a workshop to highlight work on Ukrainian language technologies.

\subsection*{Submission and Acceptance}
We had two routes to submit papers to ACL 2023: directly to the conference or through ACL Rolling Review (ARR). We received a record number of direct submissions (3601 long papers and 958 short papers) in January 2023. In addition, we received 305 commitments from ARR (271 long papers and 34 short papers) in March 2023. In total, we considered 4864 (3872 long and 992 short) papers with 70 senior area chairs, 438 area chairs, 4024 reviewers, 445 secondary reviewers, and 21 ethics reviewers in 27 tracks. We accepted 910 (23.50\%) long and 164 (16.53\%) short papers for the main conference, and 712 (41.89\% including the long papers for the main conference) long and 189 (35.58\% including the short papers for the main conference) short papers for Findings. To sum long and short papers, ACL 2023 accepted 1074 (22.08\%) papers for the conference and 901 (40.60\% including the papers for the main conference) papers for Findings.

The ACL 2023 program also features 46 papers from the Transactions of the Association for Computational Linguistics (TACL) journal, and 7 from the Computational Linguistics (CL) journal.

\subsection*{Limitations Section and Responsible NLP Checklist}

Following EMNLP 2022 and EACL 2023, we required that each submitted paper must include an explicitly named Limitations section, discussing the limitations of the work. This was to counterbalance the practice of over-hyping the take-away messages of papers, and to encourage more rigorous and honest scientific practice. This discussion did not count towards the page limit, and we asked reviewers to not use the mentioned limitations as reasons to reject the paper, unless there was a really good reason to.

In addition to the mandatory discussion of limitations, a new element at ACL 2023 is that the Responsible NLP Checklist for the accepted papers is not only considered by the reviewers, but also published together with the accepted papers as a special appendix, in an effort to improve transparency and accountability in the field.

\subsection*{Areas}
To ensure a smooth process, the submissions to ACL 2023 were divided into 26 areas. The areas mostly followed these of previous ACL, and more broadly *ACL conferences, reflecting the typical divisions in the field. Following EMNLP 2022, we split the ``Large Language Models'' track away from ``Machine learning in NLP", reflecting the growth of submissions in the area. We also offered two new tracks (``Linguistic diversity'' and ``Multilingualism and Cross-Lingual NLP''). For the papers authored by SACs, the final recommendation decisions were made by a separate SAC team. 
%We also had a special area for papers for which both SACs had a conflict of interest. Those papers were reviewed by the reviewers and ACs in their original areas, but the paper recommendations were made by a dedicated SAC, who was a senior member of the NLP community. 
The most popular areas (with over 250 submissions) were ``Dialogue and Interactive Systems'', ``Information Extraction'', ``Large Language Models'', ``Machine Learning for NLP'', and ``NLP Applications''.

\subsection*{Best Paper Awards}

% Don't think we'll get exact number in time

ACL'23 implemented the new ACL award policy, aiming to expand the pool of work that is recognized as outstanding. In total, 73 papers were nominated by the reviewers or area chairs for consideration for awards. These papers were assessed by the Best Paper Award Committee, and with their help we selected 4 best papers, 3 special awards (social impact, resource, reproduction), and several dozen outstanding papers. The best and outstanding papers will be announced in a dedicated plenary session for Best Paper Awards on July 10 2023.

% \subsection*{Programme Committee Structure and Reviewing}
% Similar to prior NLP conferences, we adopted the hierarchical program committee structure, where for each area we invited 1--3 Senior Area Chairs (SACs), who worked with a team of Area Chairs (ACs), and a larger team of reviewers. We relied on statistics from prior years to estimate how many SACs, ACs and reviewers would be needed, aiming to ensure no more than 6 papers per reviewer. We ended up with 70 SACs, 439 ACs and 4467 reviewers. For identifying ACs and reviewers, we used the reviewer lists from prior *ACL conferences, and also encouraged all ACL 2023 authors to nominate at least one qualified reviewer among the authors of the submission. The final number of review requests per reviewer candidate in our matching algorithm was determined by their expertise match to the available submissions, their declared review quotas. The SACs had the option to modify the automated assignments, and the reviewers were encouraged to reach out and report inappropriate matches.



% Rather than making assignments using a matching algorithm, we asked ACs and reviewers to bid on registered abstracts within their areas, to achieve a better fit. We went with this solution as the number of papers per area was relatively small, and we wanted to avoid poor reviewing assignments as much as possible. We then made an initial paper assignment, in which we ensured that each paper would be reviewed by at least one reviewer who bidded “yes” for the submission, and by no reviewers who bidded “no” for the submission.

% Afterwards, we asked the SACs to fine-tune the allocations, and ensure each paper had one AC and three reviewers assigned to it. 

% To ensure the review quality, we provided detailed guidelines about what reviewers should and should not do in a review, based on the EMNLP 2022 guidelines. We also asked reviewers to flag papers for potential ethical concerns.

% For pre-reviewed ARR papers, we asked SACs to not rely mainly on the reviewer scores, but to make their recommendations based on the text of the reviews, meta-reviews and the papers themselves. For making acceptance decisions, we mostly followed SAC recommendations, though also taking into account the overall quality of papers submitted to the conference. Where recommendations seemed overly harsh or lenient given the reviewers' scores, reviews, author responses, or discussions amongst reviewers, we engaged in a dialogue with the respective SACs to make the final decision about the papers in question.

% \subsection*{[MISSING DETAILS] Ethics Committee}
% Following the practice started at NAACL 2021, we formed an Ethics Committee (EC) dedicated to ethical issues. The ethics committee considered \textcolor{red}{XX} papers that were flagged by the technical reviewing committee for ethical concerns. Out of these, \textcolor{red}{XX} were conditionally accepted, meaning the ethics issues had to be addressed in the camera-ready version, to be verified by the EC prior to final acceptance, and the other 11 were accepted as is. The authors of all conditionally accepted papers submitted the camera-ready version and a short response that explained how they had made the changes requested by the EC. The EC double-checked these revised submissions and responses, and confirmed that the ethical concerns had been addressed. As a result, all conditionally accepted papers were accepted to the main conference or Findings. \textcolor{red}{any rejections purely on ethics ground?}

% \subsection*{[MISSING NUMBERS] ACL Rolling Review}
% ACL Rolling Review (ARR) is an initiative of the Association for Computational Linguistics, where the reviewing and acceptance of papers to publication venues are done in a two-step process: (1) centralized rolling review and (2) the ability to commit the reviewed papers to be considered for publication by a publication venue. ACL'23 followed EMNLP 2022 and EACL 2023 in running a process which is separate from ARR, but also allows for ARR submissions. Specifically, authors could either submit papers to ACL 2023 directly, or commit ARR reviewed papers by a certain date. We coordinated with the ARR team to extract the submission, review and meta-review from the OpenReview system, according to a submission link that the author provided when committing their ARR submission to ACL. The ARR commitment deadline was set one month before the direct submission deadline \textcolor{red}{can't remember why}. Based on ARR reviews and meta-reviews, these ARR papers were ranked by the SACs together with the direct submissions in the track. Overall, ACL had 305 papers committed from ARR, of these 129 were accepted to the main conference and 64 were accepted to Findings of ACL.

\subsection*{Presentation Mode}

In ACL 2023, there is no meaningful distinction between oral and poster presentations in terms of paper quality. The composition of the oral sessions were proposed by the SACs of their respective tracks, so as to compose a thematically coherent set of papers on a shared topic or method, which would allow for an engaging discussion. The decisions were not based on the authors' virtual or on-site attendance.


We hope you enjoy the program and the new elements we introduced (but let us know either way). We are looking forward to a great ACL 2023! \vspace{3ex}\\ 
Anna Rogers (IT University of Copenhagen, Denmark)\\
Jordan Boyd-Graber (University of Maryland, USA)\\
Naoaki Okazaki (Tokyo Institute of Technology, Japan)\\ 
ACL 2023 Programme Committee Co-Chairs